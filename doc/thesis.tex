\documentclass[draft]{article}

\usepackage{hyperref} %this should always be the last package

\newcommand{\mytitle}{Multi-thread simulation of a large scale dynamical system}
\let\oldmarginpar\marginpar
\renewcommand{\marginpar}[1]{\oldmarginpar{\raggedright #1}}

\hypersetup{
	pdfauthor={Diego Bellani},
	pdftitle={\mytitle},
	pdfsubject={Bachelor thesis},
	pdfkeywords={bachelor,thesis,programming},
	pdfproducer={LaTeX},
	pdfcreator={pdfLaTeX},
	pdfborder={0 0 0},
	pageanchor=false
}

\title{\mytitle}
\date{2021}
\author{Diego Bellani\and Enrico Tronci\thanks{Supervisor}}

\begin{document}

\begin{titlepage}
	\pagenumbering{roman}
	\maketitle

	\begin{abstract}
	(Max one page, da scrivere alla fine)
	\end{abstract}

	\iffalse
	\tableofcontents
	\listoffigures
	\listoftables
	\fi
\end{titlepage}

\pagenumbering{arabic}

\section{Introduction}\label{sec:intro}

(da scrivere alla fine)

\subsection{Context}\label{sec:context}
% Short description of context,  for example:

Fire fighting in forests is a complex activity due to the limited availability
of means and resources. Therefore the intervention strategy have to consider the
possible evolution of the fire to make the best use of what is available.

\iffalse
Lo spegnimento di un incendio boschivo è un attività complessa poichè, essendo
la disponibilità di mezzi e risorse limitata, la strategia di intervento deve in
qualche modo tenere in conto dell'evoluzione possibile dell'incendio in modo da
utilizzare la meglio le risorse disponibili.
\fi

\subsection{Motivations}\label{sec:motivations}
% Motivation for what we want to do

Describe what is missing and instead would be useful to have\dots Non ne ho la
minima idea. Also Modelica did not made the cut. The original model was
implemented in Microsoft Excel\texttrademark.
% \textregistered

\subsection{Contributions}\label{sec:contrib}

%  Describe here your contributions, namely the missing items identified in the motivations.
I have implemented from scrath the simulator using just C11 and an OpenMP
capable compiler, using only POSIX functionalities making the model viable for
large scale simulations and highly portable. % TODO: insert how big are the simulations

\subsection{Related Work}\label{sec:related_work}

% Describe here the state of the art (e.g., algorithms and tools  available). 
For each paper/tool explain what you do that it is not already available
(killing). Again no clue\dots

\subsection{Outline}\label{sec:outline}

% Give an outline of the thesis structure (one sentence per section)
In the section \ref{sec:background}\dots\\
In the section \ref{sec:methods}\dots\\
In the section \ref{sec:methods}\dots\\
In the section \ref{sec:implementation}\dots\\
In the section \ref{sec:experimental_results}\dots\\
In the section \ref{sec:conclusions}\dots\\

\section{Background}\label{sec:background}

% Put in this section all the background knowledge needed to understand what you did.
The simulator was implemented in C11 using the OpenMP compiler extension.
Because the server used had a Linux based operating system on it the C standard
library and the POSIX interfaces were used to comunicate with it. OpenMP was
chosen as the threading interface, instead of the C11 one or the POSIX one,
because of it's semplicity of use and beacuse the model was trivially
parallelizable.

A SQL database (Postgresql) was chosen to integrate all the different data
sources and to handle the concurrency. But the simulator inputs and outputs CSV
files. This format was chosen due to its simplicity and the simplicity of the
data read and written from and to disk. Another important factor was the wide
support that this format has.

\section{Methods}\label{sec:methods}

% Describe here the algorithm design from a math perspective. No code here.

\section{Implementation}\label{sec:implementation}

% describe here how you implemented the functionalities described int he methods section.
Use pseudo code or diagrams.

\section{Experimental Results}\label{sec:experimental_results}

% describe the experimental results

\subsection{Goals}\label{sec:goals}
% describe here the objectives of the experimental activity: test correctness; 
% evaluate computational performances (CPU time and RAM); ....

\subsection{Setting}\label{sec:setting}
% describe experimental setting: hardware used, software used, data used, etc. ....

\subsection{Correctness}\label{sec:correctness}

\subsection{Computational Performance}\label{sec:computational_performance}

\section{Conclusions}\label{sec:conclusions}

% Sum up what you did and outline some possible future developments .

\bibliographystyle{plain}
\bibliography{document}

\end{document}
