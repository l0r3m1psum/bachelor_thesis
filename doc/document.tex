\documentclass[draft]{article}

\usepackage[T1]{fontenc} % for italian babel warnings
\usepackage[italian]{babel}
\usepackage{hyperref} %this should always be the last package

% Abbreviazioni
\newcommand{\mytitle}{Specifica dei requisiti per la tesi}
\newcommand{\eng}[1]{\foreignlanguage{english}{#1}} %TODO: add to all english words
\let\oldmarginpar\marginpar
\renewcommand{\marginpar}[1]{\oldmarginpar{\raggedright #1}}
\newcommand{\eg}{\textit{e.g.}}
\newcommand{\psql}{\texttt{psql(1)}}
\newcommand{\file}{\textit{file}}
\newcommand{\e}{\epsilon}
\newcommand{\combvar}{Kg}
\newlength{\rulewidth}\setlength{\rulewidth}{0.4pt}
\newcommand{\myrule}{\noindent\rule{\textwidth}{\rulewidth}}

\newtheorem{requirement}{Requisito}

\hypersetup{
	pdfauthor={Diego Bellani},
	pdftitle={\mytitle},
	pdfsubject={Specifica dei requisiti},
	pdfkeywords={requisiti,tesi},
	pdfproducer={LaTeX},
	pdfcreator={pdfLaTeX},
	pdfborder={0 0 0},
	pageanchor=false
}

\title{\mytitle}
\date{2021}
\author{Diego Bellani\thanks{Studente}\and Enrico Tronci\thanks{Professore}}

\begin{document}

\begin{titlepage}
	\pagenumbering{roman}
	\maketitle

	\begin{abstract}
	Questo documento \marginpar{Da espandere.} contiene le specifiche dei
	requisiti per la tesi sulla simulazione \eng{multi-core} di incendi.
	\end{abstract}

	\tableofcontents
	\listoffigures
	\listoftables
\end{titlepage}

\pagenumbering{arabic}

\section{Introduzione}\label{sec:intro}

Questa tesi fa parte di un progetto per fornire ai vigili del fuoco uno
strumento per prevedere l'espansione di un incendio in un'area forestale. Questo
sarà fatto utilizzando dei sensori sul campo e dei satelliti per raccogliere le
informazioni. Queste ultime saranno utilizzate da un modello matematico, che
verrà realizzato da un simulatore. In fine i vigili del fuoco lo utilizzeranno
tramite un'interfaccia \eng{web}.

\section{Ambito della Tesi}\label{sec:scope}

La tesi si occuperà esclusivamente dell'implementazione del modello matematico
fornitoci da Tor Vergata \cite{mod}. Durante il documento saranno elencati tutti
i suoi requisiti che saranno etichettati ed enfatizzati per metterli in risalto.
Identifichiamo il primo

% TODO: add all errors corrected in the model

\begin{requirement}\label{thm:faster}
La simulazione deve avvenire \eng{faster-then-real-time} per poter effetuare
delle previsioni.
\end{requirement}

ed il secondo come descritto nel paragrafo \ref{sec:intro}

\begin{requirement}\label{thm:used}
Il simulatore deve poter essere utilizzato da altri programmi, in particolare da
\eng{software web}.
\end{requirement}

\section{Dati a disposizione}\label{sec:data}

Il simulatore avrà a disposizioni dati geografici da satellite \cite{cop} e
meteo da stazioni meteo sul posto, sulla zona di interesse per la simulazione.
La zona è divisa in quadrati tutti uguali e ad ognuno di questi quadrati avrà
dei dati uniformi.

Per identificare i quadrati sulla mappa essi verranno forniti con coordinate
LAEA\footnote{Proiezione Lambert azimuthal Equal-Area} Europe \cite{laeae} con
sistema di riferimento ETRS89\footnote{European Terrestrial Reference System}
\cite{etrs89}.

\paragraph{Dati Geografici}

\marginpar{Nel paragrafo \ref{sec:model} tutti questi dati sono descritti con
più precisione.}

Essi sono: l'altimetria, il tipo di foresta, il livello di urbanizzazione, il
tipo di acqua presente nella zona e la carta natura.

\paragraph{Dati Meteo}

Quelli di interesse sono solo direzione e velocità del vento.

\section{Il Modello}\label{sec:model}

Per descrivere l'evoluzione di un incendio nella foresta il modello si basa su
tre semplici osservazioni

\begin{enumerate}
\item un incendio brucia finché c'è del combustibile,
\item un incendio consuma combustibile nel tempo e
\item un incendio si può spostare in un'area limitrofa.
\end{enumerate}

Dopo aver diviso un'area rettangolare della foresta in $L^* \times W^*$ celle
quadrate, che in seguito ci permetteranno di descrivere con più precisione le
caratteristiche delle varie zone della foresta, capiamo che di ogni cella, in
riga $i$ e colonna $j$, della foresta ci interessano solo due caratteristiche:
la quantità di combustibile in essa ad un certo istante $B_{ij}(t)$ e la
presenza o meno di un incendio in essa ad un certo istante $N_{ij}(t)$, che può
assumere solo due valori 0 per niente incendio e 1 per incendio. Queste due
funzioni rappresenteranno lo \emph{stato} della cella.

Per modellare il concetto di spostamento dell'incendio in aree limitrofe si è
scelto di utilizzare un automa cellulare \cite{gol}.

Un incendio oltre che avere un singolo punto di innesco ne potrebbe avere vari
nel tempo, ad esempio nel caso di incendi dolosi. Per questo non basta una
semplice nozione di stato iniziale ma ci serve un modo per parlare
dell'\eng{input} esogeno del sistema, che chiameremo $u_{ij}(t)$, che può sempre
assumere solo due valori 0 per niente incendio e 1 per incendio.

Possiamo formalizzare tutto ciò che abbiamo detto finora con queste equazioni

\begin{eqnarray}
N_{ij}(0) &=& u_{ij}(0)\textrm{,}\\
B_{ij}(0) &=& \gamma_{ij}\textrm{,}\\
N_{ij}(t+1) &=& \cases{\max(V_{ij}(t), u_{ij}(t)), &se $\overbrace{B_{ij}(t) > 0}^\textrm{c'è combustibile}$;\cr
                       0, &altrimenti.}\\
B_{ij}(t+1) &=& \cases{\max(0, B_{ij}(t)-\beta\tau), &se $\underbrace{N_{ij}(t) > 0}_\textrm{c'è incendio}$;\cr
                       B_{ij}(t), &altrimenti.}
\end{eqnarray}

Dove $\gamma_{ij}$ è la quantità iniziale di combustibile in una cella,
$V_{ij}(t)$ è la possibilità che un l'incendio si sposti nella cella corrente da
una delle sue limitrofe (figura \ref{fig:automata}), $\beta$ è la velocità di
consumo del combustibile e $\tau$ e il passo temporale.

\begin{figure}
\centering
\setlength{\unitlength}{0.7cm}
\begin{picture}(6,6)
	\newlength{\piccenter}
	\setlength{\piccenter}{3\unitlength}
	% Grid
	\thicklines
	\multiput(0,0)(2,0){4}{\line(0,1){6}} % columns
	\multiput(0,0)(0,2){4}{\line(1,0){6}} % rows

	% Arrays
	\thinlines
	\put(\piccenter,\piccenter){\vector(1,0){2}}
	\put(\piccenter,\piccenter){\vector(0,1){2}}
	\put(\piccenter,\piccenter){\vector(-1,0){2}}
	\put(\piccenter,\piccenter){\vector(0,-1){2}}
	\put(\piccenter,\piccenter){\vector(1,1){2}}
	\put(\piccenter,\piccenter){\vector(-1,1){2}}
	\put(\piccenter,\piccenter){\vector(1,-1){2}}
	\put(\piccenter,\piccenter){\vector(-1,-1){2}}

	% Black square
	\newlength{\side}
	\setlength{\side}{0.8\unitlength}
	\linethickness{\side}
	\newlength{\ypos}
	\setlength{\ypos}{\piccenter}
	\addtolength{\ypos}{-0.5\side}
	\put(\piccenter,\ypos){\line(0,0){\side}}
\end{picture}
\caption{Automa cellulare}
\label{fig:automata}
\end{figure}

Ovviamente il cuore di questo modello è la funzione $V_{ij}(t)$, che può sempre
assumere solo due valori con la stessa semantica di $N_{ij}(t)$ e $u_{ij}(t)$,
essa è definita nel seguente modo

\begin{eqnarray}
            V_{ij}(t) &=& \max\{\,Q_{ij}(\e_1, \e_2, t) \mid (\e_1, \e_2) \in \Gamma\,\}\textrm{,}\\
               \Gamma &=& \{\,(x, y) \mid x, y \in \{-1, 0, 1\}\,\}\textrm{,}\\
Q_{ij}(\e_1, \e_2, t) &=& \cases{1, &se $p_{ij}(\e_1, \e_2, t) N_{i+\e_1j+\e_2}(t) > \theta$;\cr
                                 0, &altrimenti.}\label{eq:probtest}
\end{eqnarray}

Dove $p_{ij}(\e_1, \e_2, t)$ è la probabilità che l'incendio si sposti dalla
cella limitrofa $(i+\e_1, j+\e_2)$ alla cella corrente $(i, j)$ e $\theta$ è la
soglia di propagazione dell'incendio.

\myrule

Sospendiamo ora la nostra discussione del modelo per descrivere meglio i dati a
nostra disposizione, introdotti nel paragrafo \ref{sec:data}.

Il satellite ci mette a disposizione una serie di dati, descritti nella tabella
\ref{tab:geo}. Questi tuttavia non sono usati così come sono nel modello ma sono
``rifiniti'' con le seguenti corrispondenze matematiche.

\begin{table} % TODO: add description of data
\centering
\begin{tabular}{|c|l|c|}
	\hline
	\textbf{Simbolo} & \textbf{Nome} & \textbf{Intervallo valori}\\
	\hline
	$G$ & Foreste & 0,1,2,255\\
	$U$ & Urbanizzazione & 0,\ldots,100,255\\
	$W1$ & Water1 & 0,\ldots,4,253,255\\
	$W2$ & \underline{Water2} & 0,1\\ % NOTE: currently unused, find a way to mark it or remove it
	$P$ & Altimetria & 0,\ldots,4380\\
	\hline
\end{tabular}
\caption{Dati geografici per le singole celle.}
\label{tab:geo}
\end{table}

\marginpar{Nel paragrafo \ref{sec:implementation} vengono fatte delle
precisazioni su alcune di queste corrispondenze.}

\begin{equation}\label{eq:height}
H_{ij} = \cases{1 + G_{ij}, &se $0 \leq G_{ij} \leq 2$;\cr
                  0, &se $G_{ij} = 255$.}
\end{equation}

è l'altezza media della vegetazione nella cella,

\begin{equation}\label{eq:inhabitants}
A_{ij} = \cases{U_{ij}/100, &se $0 \leq U_{ij} \leq 100$;\cr
                0, &se $U_{ij} = 255$.}
\end{equation}

è la percentuale di abitatato nella cella,

\begin{equation}\label{eq:water}
W_{ij} = \cases{0, &se $W1_{ij} = 0$;\cr
                1, &se $W1_{ij} = 1$;\cr
                0.75, &se $W1_{ij} = 2$;\cr
                0.75, &se $W1_{ij} = 3$;\cr
                0.5, &se $W1_{ij} = 4$;\cr
                1, &se $W1_{ij} = 253$;\cr
                1, &se $W1_{ij} = 255$.}
\end{equation}

è la percentuale di acqua presente. Questi ultimi tre valori ci servono per
calcolare

\begin{equation}
S_{ij} = H_{ij} \cdot (1-A_{ij}) \cdot (1-W_{ij})\textrm{,}\\
\end{equation}

che è la percentuale di infiammabilità della cella, da notare che $A_{ij}$ la
diminuisce probabilmente perché si considerano come zone abitate quelle col
cemento.

In fine

\begin{equation}
P_{ij} = P_{ij}\textrm{,}
\end{equation}

che  è l'altezza media ed è l'unico valore che non viene cambiato.

% TODO: descrivere il comportamento del vento (gaussiano)
Mentre delle stazioni meteo sul posto ci daranno dati riduardo la velocità e la
direzione del vento\dots TODO: NON È ANCORA CHIARO COME FUNZIONANO.

\begin{eqnarray}
	F_{ij} &=& ??\\
	D_{ij} &=& ??
\end{eqnarray}

Sucessivamente nel modello saranno usate solo $S_{ij}$, $P_{ij}$, $F_{ij}$ e
$D_{ij}$.

\myrule

Ora che sappiamo quali dati abbiamo disposizione, possiamo dare una definizione
della possibilità di trasmisione dell'incendio $p_{ij}(\e_1, \e_2, t)$, della
funzione \ref{eq:probtest}.

\begin{equation}\label{eq:prob}
p_{ij}(\e_1, \e_2, t) = k_0 S_{ij} C_{i+\e_1j+\e_2}(t) d(\e_1, \e_2) f_w f_P\textrm{,}
\end{equation}

% NOTE: $k_0$ è UN parametro o IL parametro?
dove $k_0$ è un parametro di ottimizzazione di soglia e

\begin{equation}\label{eq:combust}
C_{i+\e_1j+\e_2}(t) = \sin\left(\pi\frac{B_{i+\e_1j+\e_2}(t)}{\gamma_{i+\e_1j+\e_2}}\right)\textrm{,}
\end{equation}

è lo stato di combustione della cella. Largomento di $C_{ij}(t)$ può assumere
solo valori tra 0 e $\pi$, quindi avra un grafico come quello in figura
\ref{fig:sin}, overro quando metà del carburante è stato consumato allora
l'incendio avrà la massima potenza.

\begin{figure}
\centering
\setlength{\unitlength}{1cm}
\newlength{\mylength}\setlength{\mylength}{3\unitlength}
\begin{picture}(\mylength,\mylength)(0,0)
	\multiput(.5\mylength,0)(0,.1\mylength){10}{\line(0,1){.05\mylength}}
	\put(0,0){\vector(1,0){\mylength}} % ascissas
	\put(0,0){\vector(0,1){\mylength}} % ordinates

	% Magic numbers from: https://stackoverflow.com/questions/29022438/how-to-approximate-a-half-cosine-curve-with-bezier-paths-in-svg
	\qbezier(0,0)(0.3642\mylength,\mylength)(0.5\mylength,\mylength)
	\qbezier(\mylength,0)(0.6358\mylength,\mylength)(0.5\mylength,\mylength)
\end{picture}
\caption{Grafico $\sin(x)$ tra 0 e $\pi$}
\label{fig:sin}
\end{figure}

\begin{equation}\label{eq:disom}
d(\e_1, \e_2) = \left(1-\frac{1}{2}|\e_1\e_2|\right)\textrm{,}
\end{equation}

è il fattore di fattore di disomogeneità al confine delle celle. Esso può
assumere solo due valori 1 e $1/2$, il primo nelle celle sopra, sotto e ai lati
della cella corrente, il secondo sulle celle diagonali. Ovvero l'incendio si
trasmette la metta da una cella diagonale.
% NOTE: probabilmente un disegno aiuterebbe a spiegarlo più intuitivamente.

% TODO: dare una spiegazione intuitiva della semantica di queste due funzioni.
\begin{equation}\label{eq:wind}
f_w = \exp\left(k_1 F_{i+\e_1j+\e_2}\frac{\begin{array}{c}\e_1\cos(D_{i+\e_1j+\e_2})\\
      +\\\e_2\sin(D_{i+\e_1j+\e_2})\end{array}}{\sqrt{\e_1^2 + \e_2^2}}\right)
\end{equation}

è il contributo dato dal vento, dove $k_1$ è il parametro di ottimizzazione di
questo contributo, in fine

\begin{equation}\label{eq:slope}
f_P = \exp\left(k_2\arctan\left(\frac{P_{ij}-P_{i+\e_1j+\e_2}}{L}\right)\right)
\end{equation}

è il contributo dato da il dislivello tra le due celle, dove $k_2$ è il
parametro di ottimizzazione di questo contributo.

Tutti i parametri globali del modello e delle singole celle sono stati messi
nelle tabelle riassuntive \ref{tab:globals} e \ref{tab:params}.

\begin{table}
\centering
\begin{tabular}{|c|l|c|}
	\hline
	\textbf{Simbolo} & \textbf{Nome} & \textbf{Unità di Misura}\\
	\hline
	$\tau$ & passo temporale & secondi\\
	$L$ & lato della singola cella & metri\\
	$L^*$ & lunghezza area monitorata & metri\textsuperscript{2}\\
	$W^*$ & ampiezza area monitorata & metri\textsuperscript{2}\\
	$\beta$ & consumo combustibile & \combvar/secondo\\
	$\theta$ & probailità propagazione & adimensionale\\
	$k_0$ & ottimizzazione soglia & adimensionale\\
	$k_1$ & ottimizzazione vento & adimensionale\\
	$k_2$ & ottimizzazione pendenza & adimensionale\\
	\hline
\end{tabular}
\caption{Parametri globali del modello.}
\label{tab:globals}
\end{table}

\begin{table}
\centering
\begin{tabular}{|c|l|c|}
	\hline
	\textbf{Simbolo} & \textbf{Nome} & \textbf{Unità di Misura}\\
	\hline
	$H$ & altezza media della vegetazione & metri\\
	$A$ & percentuale abitazione & adimensionale\\
	$W$ & percentuale acqua presente & adimensionale\\
	% NOTE: a quanto pare \gamma è una costante
	$\gamma$ & quantità di combustibile & \combvar\\
	$S$ & percentuale infiammabilità & adimensionale\\
	$P$ & altitudine media & metri\\
	$D$ & direzione del vento & radianti\\
	$F$ & velocità del vento & metri/secondo\\
	\hline
\end{tabular}
\caption{Parametri delle singole celle.}
\label{tab:params}
\end{table}

\section{Il Simulatore}

\marginpar{Questa sezione è da riorganizzare e riscrivere.}

Il programma dovrà simulare l'evoluzione dell'incendio dati il modello e i dati
descritti nel paragrafo \ref{sec:model}. Data la grandezza dell'area
descritta\footnote{Probabilmente questo è il termine sbagliato.} da
\cite{laeae}, il programma effetuerà la simulaizone solo su una piccola parte di
quest'ultima.

\begin{requirement}\label{thm:subrect}
Il simulatore deve poter simulare su dei sotto rettangoli di \cite{laeae} a
scelta.
\end{requirement}

Nel modello non è specificata nessuna condizione di terminazione e per quanto
sia possibile far continuare la simulazione per un tempo indeterminato è più
comodo specificare un \emph{orizonte di simulazione}.

\begin{requirement}\label{thm:horizon}
La simulazione andrà avanti fino ad un determinato orizonte $h$.
\end{requirement}

Inoltre avrebbe senso e farebbe comodo agli utenti poter vedere lo stato della
simulazione anche prima che questa termini, per poter creare cose come
animazioni.

\begin{requirement}\label{thm:snapshot}
Il simulatore deve esporre il suo stato interno ogni $s$ step, questo numero è
detto frequenza di \eng{snapshot}.
\end{requirement}

Alcuni dati, come la direzione e la forza del vento, potrebbero drasticamente
cambiare ad un certo punto della simulazione. Essendo questa
\eng{faster-then-real-time} potrebbe essere necessario riprendere la
simulzione da un certo punto nel passato\footnote{Di cui convenientemente
teniamo degli \eng{snapshot}, come descritto nel requisito \ref{thm:snapshot}.}.
Lo stesso discorso vale per la comparsa di nuovi focolai in punti non
previsti\footnote{Questa è un eventualità probabile nel caso di incendi dolosi.}
come modellato dalla funzione $u_{ij}(t)$.

\begin{requirement}
Il simulatore deve poter essere inizializzato con uno stato iniziale arbitrario.
\end{requirement}

\begin{requirement}\label{thm:termination}
Il simulatore deve poter essere fermato in maniera ``gentile''.
\end{requirement}

Per garantire la correttezza dei risultati il simulatore deve poter validare
ciò che riceve in \eng{input} e per fare ciò deve ricevere delle specifiche sul
formato dei dati, come la loro grana (i.e. quanto spazio ricopre ogni singolo
\eng{datapoint}).

\begin{requirement}\label{thm:schema}
Il dati devono essere corredati di una sorta di schema che ne specifici le
caratteristiche.
\end{requirement}

\begin{requirement}
Il simulatore deve validare i dati che riceve basandosi sugli schemi del
requisito \ref{thm:schema}.
\end{requirement}

Per come è scritto il modello se eseguito su un'area rettangolare non specifica
cosa fare sui contorni del rettangolo, ovvero cosa fare quanto intorno ad ogni
cella non ci sono 8 altre celle.

\begin{requirement}\label{thm:boundary}
Il siumulatore deve gestire la simulazione nei contorni dell'area rettangolare.
\end{requirement}

Mentre come descritto dal requisito \ref{thm:schema} i dati hanno delle grane,
non necessariamente uguali tra loro, anche il simulatore ne ha una sua interna
sia per permettere una maggiore precisione nei risultati che per permettere dati
con granularità diversa.

\begin{requirement}\label{thm:grain}
Il simulatore deve avere una sua grana interna, necessariamente minore uguale a
quella dei dati ricevuti in \eng{input}.
\end{requirement}

\subsection{Dettagli di implementazione}\label{sec:implementation}

Il simulatore girera su un \eng{kernel} Linux \cite{kern}, e sarà,
probabilmente, invocato da un servizio \eng{web}. Tutti i dati necessari per la
simulazione saranno letti da \eng{file} CSV \cite{csv}.

\begin{requirement}\label{thm:csv}
Il simulatore deve poter leggere i \eng{file} CSV.
\end{requirement}

Il sottorettangolo del requisito \ref{thm:subrect}, sarà espresso come una
coppia di coordinate $p_1$ e $p_2$ che indicheranno rispettivamente l'angolo in
basso a sinistra e l'angolo in alto a destra del rettandolo selezionato.

La terminazione ``gentile'', descritta nel requisito \ref{thm:termination} sara
effetuata alla ricezione del segnale \texttt{SIGUSR1} dal processo.

\begin{requirement}
Il programma deve poter ricevere e gestire segnali.
\end{requirement}

Il requisito \ref{thm:boundary} sarà gestito semplicemente ignorando il contorno
dell'area rettangolare, come in figura \ref{fig:boundary}, questo perché l'area
selezionata dovrebbe essere comunque abbastanza grande da non influire sul
risultato finale e per rimanere il più fedeli al modello possibile, senza
contare la semplicità di implementazione.

% TODO: 90 deg to the right to avoid the ugly hack
\begin{figure}
\centering
\setlength{\unitlength}{0.7cm}
\begin{picture}(6,6)
	% Grid
	\thicklines
	\newlength{\cellside}\setlength{\cellside}{1.5\unitlength}
	\newcommand{\cellbars}{4}
	\multiput(0,0)(\cellside,0){\cellbars}{\line(0,1){5}} % columns
	\multiput(0,0)(0,\cellside){\cellbars}{\line(1,0){5}} % rows

	\newlength{\dotslength}\settowidth{\dotslength}{\dots}
	\newlength{\dotsdist}\setlength{\dotsdist}{6\unitlength minus 0.5\cellside}
	\addtolength{\dotsdist}{-\dotslength}
	\put(\dotsdist, 0.5\cellside){\dots}
	\put(0.5\cellside, \dotsdist){\vdots}
	\put(\dotsdist, \dotsdist){$\cdot^{\cdot^\cdot}$} % Ugly hack

	\newsavebox{\cross}
	\savebox{\cross}(0.7,0.7)[bl]{
		\put(0,0){\line(1,1){1}}
		\put(1,0){\line(-1,1){1}}
	}
	\multiput(0.3,0.3)(\cellside,0){3}{\usebox{\cross}}
	\multiput(0.3,0.3)(0,\cellside){3}{\usebox{\cross}}
\end{picture}
\caption{Contorno ignorato.}
\label{fig:boundary}
\end{figure}

Oltre ai parametri nella tabella \ref{tab:globals} il simulatore ha anche
bisogno di alcuni parametri agiuntivi, descritti rispettivamente nei requisiti
\ref{thm:horizon}, \ref{thm:snapshot}, \ref{thm:subrect} e \ref{thm:grain},
riassunti nella tabella \ref{tab:config}.

\begin{table}
\centering
\begin{tabular}{|c|l|}
	\hline
	\textbf{Simbolo} & \textbf{Nome}\\
	\hline
	$h$ & orizonte di simulazione\\
	$s$ & frequanza \eng{snapshot}\\
	$p_1$ & primo punto rettangolo\\
	$p_2$ & secondo punto rettangolo\\
	$g$ & grana simulatore\\
	\hline
\end{tabular}
\caption{Parametri di configurazione addizionali del simulatore.}
\label{tab:config}
\end{table}

Per garantire una certa flessibilità e usabilità i file da leggere e la cartella
dove verrano messi gli \eng{snapshot} saranno specificati da linea di comando.

\begin{requirement}
Il simulatore deve poter leggere gli argomenti da linea di comando.
\end{requirement}

Le equazioni \ref{eq:height}, \ref{eq:inhabitants} e \ref{eq:water} presentano
casi di conversioni con i valori 255 e 253, questi son valori che nonostate
siano legali non sono desiderati nei dati, di conseguenza l'utente deve esssere
avvisato della presenza di questi valori potenzialmente indesiderati.

\begin{requirement}
Il simulatore deve poter avvertire l'utente della presenza di dati indesiderati.
\end{requirement}

Per poter realizare il requisito \ref{thm:faster} il simulatore deve sfruttare
al meglio tutto l'\eng{hardware} che gli viene messo a disposizione.

\begin{requirement}\label{thm:multicore}
Il simulatore deve poter sfruttare tutta la potenza di calcolo della macchina,
come ad esempio la parallelizzazione in un singolo \eng{core} (SIMD) e
\eng{multi-core}, perché il problema è \eng{trivially parallelizable}.
\end{requirement}

Dato che spesso la dimensione dell'area da simulare sarà considerevole,
bisognerà tenere conto della dimensione delle \eng{cache} del processore,

\begin{requirement}\label{thm:blocking}
% https://software.intel.com/content/www/us/en/develop/articles/cache-blocking-techniques.html
% https://csapp.cs.cmu.edu/public/waside/waside-blocking.pdf
Il simulatore dovra utilizzare tecniche come il \eng{cache blocking} per
migliorare la località temporale dei dati usati dal programma.
\end{requirement}

Per poter realizzare al meglio il requisito \ref{thm:blocking} è necessario
poter determinare la dimensione della \eng{cache} del processore. Per fare ciò
\texttt{x86} mette a disposizione l'istruzione \texttt{CPUID}.

\begin{requirement}
Il simulatore deve poter determinare la dimensione della \eng{cache} del
processore a tempo di esecuzione.
\end{requirement}

Come descritto dal requisito \ref{thm:used}, il simulatore deve poter essere
utilizzato da altri servizi, inoltre è necessario mantenere e trattare tutti i
dati geografici e meteorologici in maniera uniforme. Per fare ciò useremo una
base di dati.

\begin{requirement}\label{thm:db}
Il simulatore deve potersi interfacciare con una base di dati.
\end{requirement}

Per soddisfare il requisito \ref{thm:db} useremo PostgreSQL 13 \cite{psql}. Dal
requisito \ref{thm:db}, \ref{thm:csv} e \ref{thm:snapshot} ne derivano i
seguenti tre

\begin{requirement}
La base di dati ha bisogno di uno schema.
\end{requirement}

\begin{requirement}\label{thm:export}
La base di dati deve poter esportare i dati in formato CSV.
\end{requirement}

\begin{requirement}\label{thm:import}
La base di dati deve poter importare i dati dal formato CSV.
\end{requirement}

I requisiti \ref{thm:export} e \ref{thm:import} possono essere facilmete
implementati grazie alle funzionalità di PostgreSQL come mostrato nelle figure
\ref{fig:export} e \ref{fig:import}.

\begin{figure}
\centering\verb+\copy <table> from 'data.csv' with (format csv);+
\caption{Importazione dei dati nella base.}
\label{fig:import}
\end{figure}

\begin{figure}
\centering\verb+\copy <table> to 'export.csv' with (format csv);+
\caption{Esportazione dei dati dalla base.}
\label{fig:export}
\end{figure}

\bibliographystyle{plain}
\bibliography{document}

% TODO: definire bene il formato dei file di output, perché da essi dipende
% la possibilità di poter far cominciare la similazione con in input uno stato
% raggiunto in precedenza.
\iffalse
\subsubsection{Formato dei CSV}

\marginpar{Unità di misura dei dati? Tipo dei dati (int o float)?}

I CSV emessi dalla base di dati conterranno in ogni riga i dati che si
riferiscono ad una singola cella. I \file\ che ne indicano la grana avranno
invece una struttura come quella della tabella \ref{tab:grain}, dove $x_n$ e
$y_n$ rappresentano le coordinate degli angoli dell'area di interesse mentre
\eng{length} e \eng{width} rappresentano, rispettivamente, la lunghezza e l
larghezza di ogni singola cella.

\begin{table}
\centering
\begin{tabular}{|c|c|c|c|c|c|}
	\hline
	$x_1$ & $y_1$ & $x_2$ & $y_2$ & \textbf{\eng{length}} & \textbf{\eng{width}}\\
	\hline
	450070 & 12980 & 670010 & 13300 & 1 & 1\\
	\hline
\end{tabular}
\caption{Esempio \eng{file} grana.}
\label{tab:grain}
\end{table}

Quelli emessi dal simulatore invece avranno una struttura come quella della
tabella \ref{tab:output}.

\begin{table}
\centering
\begin{tabular}{|c|c|c|c|c|}
	\hline
	$n_{11}$ & $b_{11}$ & \dots & $n_{1i}$ & $b_{1i}$\\
	\vdots & \vdots & $\ddots$ & \vdots & \vdots\\
	$n_{j1}$ & $b_{j1}$ & \dots & $n_{ji}$ & $b_{ji}$\\
	\hline
\end{tabular}
\caption{Esempio \eng{output} simulatore.}
\label{tab:output}
\end{table}

Dove $n$ è lo stato della cella, in fiamme o meno, e $b$ è la sua quantità di
combustibile ancora a disposizione.
\fi

\iffalse % Old stuff for updating the data mid simulation
I dati saranno quindi presi in \eng{input} da tre \file\ CSV: uno per i dati
geografici, uno per quelli meteorologici e uno per lo stato iniziale
dell'incendio. Ognuno di questi file sarà corredato da un'altro CSV che ne
conterrà le informazioni sulla ``grana''.

Per permettere l'aggiornamento di questi dati durante la simulazione il
simulatore monitorerà la presenza di \file\ con nomi come \texttt{land.csv},
\texttt{meteo.csv} e \texttt{fire.csv}. Nel caso in cui uno di questi file sia
presente, dopo l'inizio della simulazione il simulatore provvederà a leggerli,
aggiornare il suo stato interno e cambiargli nome\footnote{con nomi tipo
\texttt{land\_<timestamp>.csv}} (o eliminarli) per evitare di rileggerli in
futuro.

Un implementazione ingenua della scrittura di questi file per l'aggiornamento
dei dati potrebbe generare una \eng{race-condition}, ovvero la lettura di
incompletidati, dato che la scrittura in un file non è atomica. Per evirate ciò
una strateggia potrebbe essere quella di scrivere i dati in file con nomi tipo
\texttt{tmpXXX} e una volta finito di scriverli si può usare \texttt{rename(2)},
che è atomica, per dargli il nome giusto. Questo metodo sarà anche usata per la
scrittura dei file.
\fi

\end{document}
