\documentclass[draft]{article}

\usepackage[T1]{fontenc} % for italian babel warnings
\usepackage[italian]{babel}
\usepackage{hyperref} %this should always be the last package

% Abbreviazioni
\newcommand{\mytitle}{Specifica dei requisiti per la tesi}
\newcommand{\eg}{\textit{e.g.}}
\newcommand{\psql}{\texttt{psql(1)}}
\newcommand{\file}{\textit{file}}
\newcommand{\e}{\epsilon}

\newtheorem{requirement}{Requisito}

\hypersetup{
	pdfauthor={Diego Bellani},
	pdftitle={\mytitle},
	pdfsubject={Specifica dei requisiti},
	pdfkeywords={requisiti,tesi},
	pdfproducer={LaTeX},
	pdfcreator={pdfLaTeX},
	pdfborder={0 0 0},
	pageanchor=false
}

\title{\mytitle}
\date{2021}
\author{Diego Bellani\thanks{Studente}\and Enrico Tronci\thanks{Professore}}

\begin{document}

\begin{titlepage}
	\maketitle

	\begin{abstract}
	Questo documento \marginpar{Da espandere.} contiene le specifiche dei
	requisiti per la tesi sulla simulazione \textit{multi-core} di incendi.
	\end{abstract}

	\tableofcontents
	% \listoffigures
	% \listoftables
\end{titlepage}

\section{Introduzione}

Questa tesi fa parte di un progetto per fornire ai vigili del fuoco uno
strumento per prevedere l'espansione di un incendio in un'area forestale. Questo
saraà fatto utilizzando dei sensori sul campo e dei satelliti per raccogliere le
informazioni. Queste ultime saranno utilizzate da un modello matematico, che
verrà realizzato da un simulatore. In fine i vigili del fuoco lo utilizzeranno
tramite un'interfaccia web.

\section{Ambito della Tesi}

La tesi si occuperà esclusivamente dell'implementazione del modello matematico
fornitoci da Tor Vergata. Durante il documento saranno elencati tutti i suoi
requisiti. Identifichiamo il primo.

\begin{requirement}
La simulazione deve avvenire \textit{faster-then-real-time} per poter effetuare
delle previsioni.
\end{requirement}

\section{Dati a disposizione}

Il simulatore avrà a disposizioni dati geografici e meteo, da satelliti, sulla
zona di interesse per la simulazione. La rona è divisa in rettangoli tutti
uguali e ad ognuno di questi rettangoli avrà dei dati uniformi.

Per identificare i rettangoli sulla mappa essi verranno forniti con coordinate
LAEA\footnote{proiezione Lambert azimuthal Equal-Area} Europe \cite{laeae} con
sistema di riferimento ETRS89\footnote{European Terrestrial Reference System}
\cite{etrs89}.

\paragraph{Dati Geografici}

\marginpar{Più avanti tutti questi dati saranno descritti con più precisione.}

Essi sono: l'altimetria, il tipo di foresta, il livello di urbanizzazione, il
tipo di acqua presente nella zona e la carta natura.

\paragraph{Dati Meteo}

Quelli di interesse sono solo direzione e velocità del vento.

\section{Il Modello}

Il modello matematico che descrive l'evoluzione dell'incendio in un'area
forestale è uno di tipo automa cellulare come quello in figura
\ref{fig:automata}.

\subsection{Statica}

Il modello ha dei parametri che sono costanti e uguali per tutte le celle,
descritti nella tabella \ref{tab:globals} ed altri, sempre costanti, che possono
cambiare di cella in cella descritti nella tabella \ref{tab:params}. Questi
ultimi sono derivati dai dati geografici descritti nella tabella \ref{tab:geo}
con le seguenti corrispondenze matematiche:

\marginpar{In seguito saranno fatte delle precisazioni su alcune di queste
corrispondenze.}

\begin{eqnarray}
H_{ij} &=& \cases{1 + G_{ij}, &se $0 \leq H_{ij} \leq 2$;\cr
                  0, &se $H_{ij} = 255$.}\\
A_{ij} &=& \cases{U_{ij}/100, &se $0 \leq U_{ij} \leq 100$;\cr
                  0, &se $U_{ij} = 255$.}\\
W_{ij} &=& \cases{0, &se $W1_{ij} = 0$;\cr
                  1, &se $W1_{ij} = 1$;\cr
                  0.75, &se $W1_{ij} = 2$;\cr
                  0.75, &se $W1_{ij} = 3$;\cr
                  0.5, &se $W1_{ij} = 4$;\cr
                  1, &se $W1_{ij} = 253$;\cr
                  1, &se $W1_{ij} = 255$.}\\
S_{ij} &=& H_{ij} \cdot (1-A_{ij}) \cdot (1-W_{ij})\textrm{,}\\
P_{ij} &=& P_{ij}\textrm{.}
\end{eqnarray}

\begin{figure}
\centering
\setlength{\unitlength}{0.7cm}
\begin{picture}(6,6)
	\newlength{\piccenter}
	\setlength{\piccenter}{3\unitlength}
	% Grid
	\thicklines
	\multiput(0,0)(2,0){4}{\line(0,1){6}} % columns
	\multiput(0,0)(0,2){4}{\line(1,0){6}} % rows

	% Arrays
	\thinlines
	\put(\piccenter,\piccenter){\vector(1,0){2}}
	\put(\piccenter,\piccenter){\vector(0,1){2}}
	\put(\piccenter,\piccenter){\vector(-1,0){2}}
	\put(\piccenter,\piccenter){\vector(0,-1){2}}
	\put(\piccenter,\piccenter){\vector(1,1){2}}
	\put(\piccenter,\piccenter){\vector(-1,1){2}}
	\put(\piccenter,\piccenter){\vector(1,-1){2}}
	\put(\piccenter,\piccenter){\vector(-1,-1){2}}

	% Black square
	\newlength{\side}
	\setlength{\side}{0.8\unitlength}
	\linethickness{\side}
	\newlength{\ypos}
	\setlength{\ypos}{\piccenter}
	\addtolength{\ypos}{-0.5\side}
	\put(\piccenter,\ypos){\line(0,0){\side}}
\end{picture}
\caption{Automa cellulare}
\label{fig:automata}
\end{figure}

\begin{table}
\centering
\begin{tabular}{|c|l|c|}
	\hline
	\textbf{Nome parametro} & \textbf{Descrizione} & \textbf{Unità di Misura}\\
	\hline
	$\tau$ & passo temporale & secondi\\
	$L$ & lunghezza della singola cella & metri\\
	$W$ & ampiezza della singola cella & metri\\
	$L^*$ & lunghezza area monitorata & celle\\
	$W^*$ & ampiezza area monitorata & celle\\
	$\beta$ & consumo combustibile & combustibile/secondo\\
	$\theta$ & probailità propagazione & adimensionale\\
	\hline
\end{tabular}
\caption{Parametri globali della simulazione.}
\label{tab:globals}
\end{table}

\begin{table}
\centering
\begin{tabular}{|c|l|c|}
	\hline
	\textbf{Nome Dato} & \textbf{Descrizione} & \textbf{Intervallo valori}\\
	\hline
	$G_{ij}$ & Foreste & 0,1,2,255\\
	$U_{ij}$ & Urbanizzazione & 0,\ldots,100,255\\
	$W1_{ij}$ & Water1 & 0,\ldots,4,253,255\\
	$W2_{ij}$ & Water2 & 0,1\\
	$P_{ij}$ & Altimetria & 0,\ldots,4380\\
	\hline
\end{tabular}
\caption{Dati geografici per le singole celle.}
\label{tab:geo}
\end{table}

\begin{table}
\centering
\begin{tabular}{|c|l|}
	\hline
	\textbf{Nome parametro} & \textbf{Descrizione}\\
	\hline
	$H_{ij}$ & altezza media della vegetazione\\
	$A_{ij}$ & abitazione media\\
	$W_{ij}$ & presenza di acqua\\
	$\gamma_{ij}$ & quantità di combustibile\\ % non è chiaro come calcolarlo
	$S_{ij}$ & indice di infiammabilità\\
	$P_{ij}$ & altitudine media\\
	$D_{ij}$ & direzione del vento\\
	$F_{ij}$ & velocità del vento\\
	\hline
\end{tabular}
\caption{Parametri delle singole celle.}
\label{tab:params}
\end{table}

\subsection{Dinamica}

Ora tutta la parte statica del modello è stata descritta e possiamo passare alla
sua dinamica. Questa è descritta tramite tre funzioni diopendenti descritte
nella tabella \ref{tab:dynamic}.

Le funzione in quest'ultima tabella sono definite come segue

\marginpar{Definirla una probabilità è incorretto siccome la funzione è
deterministica.}

\begin{equation}
p_{ij}(\e_1, \e_2, t) = k_0 S_{ij} C(i+\e_1, j+\e_2) d(\e_1, \e_2) f_w f_P\textrm{,}
\end{equation}

dove $k_0$ è un parametro di ottimizzazione di soglia e

\begin{eqnarray}
(\e_1, \e_2) &\in& \Gamma = \{\,(x, y) \mid x, y \in \{-1, 0, 1\}\,\}\textrm{,}\\
C(i,j) &=& \sin\left(\pi\frac{B_{ij}}{\gamma_{ij}}\right)\textrm{,}\label{eq:combust}\\
d(\e_1,\e_2) &=& \left(1-\frac{1}{2}|\e_1\e_2|\right)\textrm{,}\label{eq:disom}\\
f_w &=& \exp\left(k_1 F(i+\e_1, j+\e_2)\frac{\begin{array}{c}\e_1\cos(D(i+\e_1,j+\e_2))\\
        \mbox{}+\e_2\sin(D(i+\e_1,j+\e_2))\end{array}}{\sqrt{\e_1^2 + \e_2^2}}\right)\textrm{,}\label{eq:wind}\\
f_P &=& \exp\left(k_2\arctan\left(\frac{P_{ij}-P_{i+\e_1j+\e_2}}{L}\right)\right)\textrm{.}\label{eq:slope}
\end{eqnarray}

L'equazione \ref{eq:combust} indica lo stato di combusione della cella, la
\ref{eq:disom} è il fattore di disomogeneità, mentre \ref{eq:wind} e
\ref{eq:slope} sono il contributo alla probabilità del vento e della pendenza
rispettivamente.

Nell'equazione \ref{eq:wind} $k_1$ è un parametro di ottimizzazione relativo al
contributo vento, nell'equazione \ref{eq:slope} $k_2$ è il parametro di
ottimizzazione relativo al contributo pendenza.

Sia

\begin{equation}
Q_{ij}(\e_1, \e_2, t) = \cases{1, &se $p_{ij}(\e_1, \e_2) N_{i+\e_1j+\e_2} > \theta$;\cr
                               0, &altrimenti.}
\end{equation}

Quindi $Q_{ij}(\e_1, \e_2, t) = 1 \iff N_{i+\e_1j+\e_2} = 1 \land
p_{ij}(\e_1, \e_2) > \theta$, in questo caso l'incendio nella cella limitrofa
$(i+\e_1, j+\e_2$ si trasferisce nella cella $(i,j)$. Per controllare tutte le
celle limitrofe ad una useremo la seguente formula

\begin{equation}
V_{ij}(t) = \max\{\,Q_{ij}(\e_1, \e_2, t) \mid (\e_1, \e_2) \in \Gamma\,\}\textrm{,}
\end{equation}

la quale è 1 quando una cella limitrofa a $(i, j)$ porta l'incendio nella cella
stessa.

\marginpar {Ovviamente l'input esogeno può cambiare nel tempo.}

In fine sia $u_{ij}$ l'input esogeno che definisce o lo stato di quete o di
incendio con i valori 0 e 1 rispettivamente, che all'atto pratico è ottenuta
attraverso dati sul campo. La dinamica del sistema è la seguente:

\begin{eqnarray}
N_{ij}(0) &=& u_{ij}(0)\textrm{,}\\
B_{ij}(0) &=& \gamma_{ij}\textrm{,}\\
N_{ij}(t+1) &=& \cases{\max(V_{ij}(t), u_{ij}(t)), &se $B_{ij}(t) > 0$;\cr
                       0, &altrimenti.}\\
B_{ij}(t+1) &=& \cases{\max(0, B_{ij}(t)-\beta\tau), &se $N_{ij}(t) > 0$;\cr
                       B_{ij}(t), &altrimenti.}
\end{eqnarray}

\begin{table}
\centering
\begin{tabular}{|c|l|}
	\hline
	\textbf{Nome parametro} & \textbf{Descrizione}\\
	\hline
	$B_{ij}(t)$ & combustibile disponibile\\
	$N_{ij}(t)$ & stato (in fiamme o meno)\\
	$p_{ij}(\epsilon_1, \epsilon_2, t)$ & probabilità di trasmissione\\
	\hline
\end{tabular}
\caption{Parametri delle singole celle che variano nel tempo.}
\label{tab:dynamic}
\end{table}

% TODO: scrivere come si gestiscono le "condizioni limite"
% TODO: avvertire il prof della conversione fatta a 1 invece che a 0 (o quello che era)
% TODO: il vento è gaussiano

\section{Il Simulatore}

Da qua in poi è tutto da riscrivere più o meno

\noindent\hrulefill

L'obbiettivo della tesi è scrivere un programma che simuli efficentemente il
modello descritto sopra. Per fare ciò esso avrà bisogno di dati meteorologici e
geografici, oltre che dei punti di partenza dell'incendio e di alcuni parametri
di configurazione come l'orizonte di simulazione.

I dati sullo stato dell'incendio e del meteo potrebbero cambiare nel mentre la
simulazione è in corso. Il primo potrebbe succedere ad esempio nel caso di
incendi dolosi se un piromane dovesse appiccare nuovi focolai, il secondo
invece a causa del naturale evolvere del meteo e dati come la presenza o medo di
pioggia e la direzione del vento potrebbero cambiare drasticamente la nostra
simulazione.

\begin{requirement}
Il simulatore deve poter aggiornare i dati in tempo reale.
\end{requirement}

\subsection{Dettagli di implementazione}

I dati saranno quindi presi in input da tre \file\ CSV: uno per i dati
geografici, uno per quelli meteorologici e uno per lo stato iniziale
dell'incendio.  Ognuno di questi file sarà corredato da un'altro CSV che ne
conterrà le informazioni sulla ``grana''. I parametri di configurazione saranno
presi da linea di comando.

Dato che ogni CSV potrebbe avere una ``grana'' diversa il simulatore dovra
essere configurato con una sua dimensione di cella interna che deve essere
sempre minore uguale della minima grana dei tre \file.

\begin{requirement}
Il simulatore deve poter leggere i CSV.
\end{requirement}

Per permettere l'aggiornamento di questi dati durante la simulazione il
simulatore monitorerà la presenza di \file\ con nomi come \texttt{land.csv},
\texttt{meteo.csv} e \texttt{fire.csv}. Nel caso in cui uno di questi file sia
presente, dopo l'inizio della simulazione il simulatore provvederà a leggerli,
aggiornare il suo stato interno e cambiargli nome\footnote{con nomi tipo
\texttt{land\_<timestamp>.csv}} (o eliminarli) per evitare di rileggerli in
futuro.

\begin{requirement}
Il simulaotre deve poter monitorare la presenza di file.
\end{requirement}

Un implementazione ingenua della scrittura di questi file per l'aggiornamento
dei dati potrebbe generare una \textit{race-condition}, ovvero la lettura di
incompletidati, dato che la scrittura in un file non è atomica. Per evirate ciò
una strateggia potrebbe essere quella di scrivere i dati in file con nomi tipo
\texttt{tmpXXX} e una volta finito di scriverli si può usare \texttt{rename(2)},
che è atomica, per dargli il nome giusto. Questo metodo sarà anche usata per la
scrittura dei file.

I risultati della simulazione saranno scritti, con una frequenza scelta
dall'utente, in un \file\ CSV.

\subsubsection{Parametri da linea di comando (non è vero sono sempre passati da
file CSV)}

\begin{itemize}
\item $\tau$ il passo di tempo per la simulazione
\item $h$ orizonte di simulazione
\item $s$ frequenza degli \textit{snapshot} dello stato della simulazione
\item $l$ lunghezza in metri della singola cella (interna al simualtore)
\item $w$ larghezza in metri della singola cella (interna al simulatore)
\item $p_1$ primo punto del rettancolo selezionato sulla mappa
\item $p_2$ secondo punto del rettancolo selezionato sulla mappa
\item $\theta$ la soglia di propagazione dell'incendio
\item $\beta$ la velocità di propagazione dell'incendio
\item $\alpha$ costante di conversione del compustibile (nostra pezza in attesa
di Tor Vergata)
\item $k_0$ ottimizzazione di soglia;
\item $k_1$ ottimizzazione del contributo del vento;
\item $k_2$ ottimizzazione del contributo dell'inclinazione;
\end{itemize}

\subsubsection{Formato dei CSV}

I CSV emessi dalla base di dati conterranno in ogni riga i dati che si
riferiscono ad una singola cella. I \file\ che ne indicano la grana avranno
invece la seguente struttura:

\footnote{Domanda}{unità di misura dei dati e tipo (int o float)?}
\begin{verbatim}
x1,y1,x2,y2,length,width
450070,12980,670010,13300,1,1
\end{verbatim}

dove \texttt{xn} e \texttt{yn} rappresentano le coordinate degli angoli
dell'area di interesse mentre \texttt{length} e \texttt{width} rappresentano,
rispettivamente, la lunghezza e l larghezza di ogni singola cella.

Quelli emessi dal simulatore invece avranno la seguente struttura:

\begin{verbatim}
n11,b11,...,n1i,b1i
...
nj1,bj1,...,nji,bji
\end{verbatim}

Dove \texttt{n} è lo stato della cella, in fiamme o meno, e \texttt{b} è la sua
quantità di combustibile ancora a disposizione.

\section{Isolamento del simulatore dall'esterno}

Per isolare il simulatore dall'esterno useremo una base di dati, in particolare
PostgreSQL, che ci permetterà di gestire i vari dati di cui la simulazione ha
bisogno in maniera uniforme.

\subsection{Creazione dello schema per la base di dati}

Questo programma deve creare la base di dati, tabelle con vincoli, \textit{view}
sui dati e vincoli per mantenerne la consistenza. Questo può essere realizzato
tramite un file SQL che verrà interpretato da \psql.

\subsection{Importazione dei dati}
\label{sec:import}

Questo programma deve importare dati nella base di dati, principalmente CSV, un
esempio di come si potrebbe fare da \psql\ è il seguente:

\begin{verbatim}
\copy <table> from 'data.csv' with format csv;
\end{verbatim}

\subsection{Esportazione dei dati}

In maniera molto simile all'importazione (sezione \ref{sec:import}) i dati
possono essere esportati grazie a \psql.

\begin{verbatim}
\copy <table> to 'export.csv' with format csv;
\end{verbatim}

\bibliographystyle{plain}
\bibliography{document}

\end{document}
